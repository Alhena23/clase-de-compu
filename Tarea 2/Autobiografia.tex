\documentclass[12pt, letterpaper]{article}
\usepackage[utf8]{inputenc}
\usepackage[spanish]{babel}
\usepackage{xcolor}

\title{¿Quién es Alexa?}
\author{Alexandra González de la Fuente  }
\date{Lunes 13 de September del 2022}

\begin{document}

\maketitle

\section{\tt\huge{Academia}}

\subsection{\tt{Pasado}}
\tt{Estudiaba en la Escuela Nacional Preparatoria \#8, que queda muy cerca del metro barranca del muerto, la cual quedaba como a 10-15 min de mi casa. Para llegar debía salir de mi casa, caminar derecho y bajar al Costco de San Antonia enfrente de periférico que es donde he vivido los últimos 9 años. Ahí en el Costco quedaba la parada y tomaba el camión con dirección "Puerta grande” que me dejaba a una calle de la prepa. Iba en la tarde así que, me salía a la 1:30.}

\subsection{\tt{Actualidad}}
\tt{Afortunadamente la Facultad queda cerca de mi casa, queda como a unos 30 min. en carro y 1 hora en transporte. Mi papá me lleva y me trae todos lo días excepto los viernes que me voy y regreso sola. Para llegar debo tomar un camión en la esquina de mi calle que me deja a una calle del Metrobús Nápoles, de ahí son como 13 estaciones hasta la estación de C.U y ya solo camino hasta la Facultad.}

\subsection{\tt{Física}}
\tt{Okey aquí viene la pregunta que he evitado desde que entré jaja, pero bueno. En la secundaria era buena en física y el profesor que me impartía la materia era apasionado, despumes en la prepa ni siquiera debía estudiar para los exámenes. Durante toda mi vida me han gustado e interesado muchas carreras pero estaba muy indecisa porque no había una que realmente me llenara y apasionara. Así que entre todas la que más me interesaba era la física, poder dedicarme a la astronomía y encontrar el porque de muchas cosas lejanas que no tomamos en cuenta muchas veces. Y bueno aquí estamos, esperando que todo salga bien y nos guste. }

\section{\textit\huge{Hobbies}}
\subsection{\textit{Leer}}

\textit{Mi cosa favorita en este mundo es leer, durante la pandemia empece a leer un libro que hace tiempo me habían comprado y cuando me di cuenta en una semana lo había terminado. El año pasado leí aproximadamente 40 libros, sobre fantasía, amor, acción, ciencia ficción…me entretenía mucho durante la cuarentena y se volvió mi cosa favorita. Me gusta porque es cómo vivir mil vidas a través de personajes, protagonistas, villanos y lugares diferentes. Actualmente solo le dedico como 5 horas a la semana.}

\subsection{\textit{Ver series}}

\textit{En mis tiempos libres me gusta ver series, me gusta mucho la series que son largas. Siento que es lindo ver crecer a los personajes a lo largo de los años, como van madurando, cambiando, creciendo…Hace que te encariñas e identificas con los personajes. A la semana le dedico como 2 horas jaja las veo mucho al hacer la tarea.}

\section{\textsl{Música}}

\subsection{\textsl{Indie Rock}}
\textsl{\tiny{Me gusta escucharla cuando estoy triste o muy feliz, en el carro o al bañarme. 
The Neighbourhood - Flawless.     Arctic Monkeys - I wanna be yours}}

\subsection{\textsl{Alternativo}}
\textsl{\tiny{Disfruto mucho escucharla cuando estoy estudiando o trato de concentrarme en algo, me relaja mucho. 
Aurora - Runaway   Girl in Red - Bad idea}}

\section{Viajes}
\subsection{Italia}
Siempre me ha parecido un lugar muy bonito y lindo, su idioma también me llama  mucho la atención. En las fotos siempre se ven vistas incréibles, con arquitectura de la epoca tan linda y difente a la que estoy acostumbrada. 
\subsection{Japón}
Visitar este país ha sido mi sueño desde que tengo 10 años. Siempre he sentido mucha fascinación por toda su cultura, religión y costumbres. En verdad creo que es un país tan único y tan diferente. \\



{\color{purple}Respecto a los libros, creo que todos deberían leer lo que les guste. Una de las razones por la que la gente odia leer es porque toda su vida les han obligado a leer y cuando tratan de enseñar}
{\color{green}el hábito de la lectura pues les ponen libro que están fuera de sus intereses. Mucjos pienan que leer se resume en, libros clásicos. Pero existen miles de géneros y todos son válidos}


\end{document}
