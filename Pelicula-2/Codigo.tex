\documentclass[a5 paper,11 pt]{article}
\usepackage[utf8]{inputenc}
\usepackage[spanish]{babel}
\usepackage[usenames]{xcolor}
\usepackage{tcolorbox}
\usepackage{graphicx}
\pagecolor{pink}
\usepackage{geometry}
\geometry{left=3cm, right=2.5cm, top=2cm}
\usepackage{fancyhdr}

\pagestyle{fancy}

\lhead{\footnotesize{Alexandra González de la Fuente}}



\begin{document}


\begin{center}
    \tt{\textbf{\Huge{La Sociedad de los poetas muertos}}}
\end{center}

\tt{
\section{¿De qué trata?}

La película empieza con la llegada del nuevo profesor de literatura de la prestigiosa Academia Welton en 1959, donde todos ahí serán grandes doctores, abogados o banqueros en un futuro. Este nuevo profesor llamado Keating ex alumno de esta misma institución, les enseña a sus alumnos el significado del tiempo y de saber aprovecharlo, como lo es el concepto de CARPE DIEM, basando sus métodos de enseñanza en la libertad del pensamiento. Siendo así todo lo contrario a lo que la institución enseña. Invitándolos a descubrir sus sueños y luchar por ellos, a no ser como lo demás y vivir su vida como ellos quieran. 
La película se centra en 6 alumnos en específico que al verse interesados y asombrados por la mentalidad de su nuevo profesor, fundan “La sociedad de los poetas muertos”, descubriendo así nuevos talentos e intereses y tomando el control de estos. Como lo es con Neil Perry uno de los alumnos que descubre su vocación por el teatro, pero al verse desafiado por su padre tendrá que tomar un decisión entre vivir libremente sus sueños o seguir bajo el control de sus padres. \textcolor{purple}{Una película que te enseñará que tener el control de tu vida y desafiar todo lo que te han enseñado, no siempre es fácil. }

\subsection{Actores}
\begin{enumerate}


\item \fcolorbox{yellow}{blue}{\textcolor{red}{Robin Williams - Personaje principal} }
\item \fcolorbox{yellow}{blue}{\textcolor{red}{Ethan Hawke - Personaje principal}}
\item \fcolorbox{yellow}{blue}{\textcolor{red}{Josh Charles - Personaje principal}}
\item \fcolorbox{yellow}{blue}{\textcolor{red}{Robert Sean Leonard - Personaje principal}}
\item \fcolorbox{yellow}{blue}{\textcolor{red}{Gale Hansen - Personaje principal}}
\item Dylan Kussman - Personaje secundario 
\item Allelon Ruggiero - Personaje secundario
\item James Waterston - Personaje secundario
\item Dir. Gale Nolan - Personaje secundario
\item Sr. Tom Perry - Personaje secundario
\item Peter Weir - Director
\item Sandy Veneziano - Dirección artística
\item Steven Haft, Paul Junger Witt, Tony Thom- Producción
\item Wendy Stites - Diseño de producción 

\end{enumerate}

\subsection{Personajes}
\begin{itemize}

     \item[*] \fcolorbox{green}{purple}{\textcolor{yellow}{John Keating}} - - Ex alumno de la Academia Welton que llega como el nuevo profesor de literatura. Un personaje con mentalidad abierta e incitando a pensar con libertad a los jóvenes. 
    \item[*] \fcolorbox{green}{purple}{\textcolor{yellow}{Neil Perry}} - Personaje principal que lleva toda su vida en Welton y que se dedicará a lo mismo que su padre. Líder que reabrió la sociedad de los poetas muertos, convenciéndolos de compartir sus más profundos secretos en compañía de toso. Un soñador que encuentra su pasión en el teatro. 	
    \item[*] \fcolorbox{green}{purple}{\textcolor{yellow}{Todd Anderson}} - Joven adolescente que acaba de entrar a la Academia Welton, compañero de Neil Perry. Se integra al grupo de amigos en la Academia y se ve interesado en todo lo nuevo que les enseña el profesor.
    \item[*]\fcolorbox{green}{purple}{\textcolor{yellow}{Knox Overstreet}} - Se enamora perdidamente de la hija de un amigo de su padre. Desafia las reglas al intentar ir al escuela de la chica que le gusta y se pelea con su novio. 	
    \item[*] \fcolorbox{green}{purple}{\textcolor{yellow}{Charles Dalton}} - El mas rebelde de todo el grupo, le gusta desafiar a los directores, un amigo leal e interesado por los sueños de sus amigos. Se une al club para ser el mismo y se autodenomina “Nuwanda”
    \item[*]Steven Meeks - Personaje secundario que siempre se preocupa por sus compañeros.
    \item[*]Richard Cameron - Cameron en un principio fue fiel  a sus compañeros, todos guardaban el secreto del club de los poetas muertos. Pero fue él quien al verse acorralado con la situación y llevando el código de honor de la escuela  delato a todos sus compañeros con el director.
    \item[*]Gerard Pitts - Gusta de la poesía y sin ningún problema decidio entrar al club de los poetas muertos, apoyando a sus demás compañeros.	
    \item[*]Director Gale Nolan - Director con viejos ideales, aferrado a sus pensamientos y juzgador de la libertad.	
    \item[*]Mr. Perry 	- Padre de Neil, con mano dura y tratando de que su hijo siga su ejemplo
    
    
\end{itemize}

\begin{figure}
 \caption{Escena de la pelicula donde el profesor les relata la libertad de se un pensador}
   \raggedleft
    \includegraphics[scale=0.2, angle=-15]{Escena de la pelicula.JPG}
    
    
\end{figure}


\section{Mi opinión }
Elegí esta película porque me enseñó muchas cosas. Empezando por esa libertad de pensamiento que le enseña su profesor, el valor real del tiempo, en tener presente que si quieres algo debes tomarlo y no debes dejar todo hasta el final. Como dice una frase \textbf{\emph{\textcolor{purple}{Abandonar todo lo que no era la vida, para no porque no descubrir en el momento de mi muerte que no había vivido}}}, y es que todo lo que les enseña se puede resumir como un vivan, vivan porque vida solo hay una y es suya; vivan con sus propias reglas, con sus propios sueños y metas…Aparte la película llegó en el momento correcto de la pandemia, pues la vida de todos dio un gran giro y nos hizo replantearlos el como estábamos viviendo y que estábamos logrando. 
También que me identifique con los personajes que son aproximadamente de mi edad, como se sienten sofocados por sus padres y que solo quisieran \textcolor{purple}{vivir su vida bajo sus propios términos y como Neil, vivir haciendo lo que aman.} Que muchas veces llegamos a sentirnos tan sofocados y abrumados que al no poder hacer esas pasión que encontraste, le quitan todo el sentido a la vida. 
En general la película te hace pensar y plantea que es tu vida y debes de vivirla feliz. }


\begin{figure}

    \raggedleft
    \includegraphics[scale=0.3,angle=-18]{Neil perry.jpg} \caption{Neil perry, que lucha por sus sueños y al verse estos arrancados y prohibidos. Toma la desición de irse.}
               
    \end{figure}

\begin{figure}
 
   \raggedleft
    \includegraphics[scale=0.3,angle=8]{Mr. Perry.JPG}
    \caption{El padre del protagonista, que lo corrompe y obliga a seguir sueños que el no desea}
    

\end{figure}



 
\end{document}
